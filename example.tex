\documentclass{curriculumvitae}

\name{John}
\surname{Smith}

\prename{}
\postname{, M.Sc.}

\residence{Berlin}
\nationality{German}

\mail{contact@johnsmith.tld}
\phone{+49.171.5556789}
\web{johnsmith.tld}
\github{johnsmith}

\begin{document}

	\cvhead

	\cvsection{\faGraduationCap}{Education}
	\cvdatedentry{Berlin University}{Mar. 2021 - Sep. 2022}{%
			M.Sc. in Mechanical Engineering (Grade: 1.2) \\
			Thesis: \textit{Advanced Methods for Plasticity Analysis}
	}
	\cvdatedentry{Berlin University}{Oct. 2017 - Mar. 2021}{%
			B.Eng. in Mechanical Engineering (Grade: 1.7) \\
			Thesis: \textit{Simulation of Structural Responses in Complex Geometries}
	}
	\cvdatedentry{Albert Einstein High School}{until Jun. 2017}{%
			General qualification for university entrance \\
			Focused on Mathematics, Physics, and Computer Science.
	}

	\cvsection{\faStore*}{Professional Experience}
	\cvdatedentry{Berlin University, Researcher in Computational Engineering}{Sep. 2022 - Present}{%
			Conducting research in non-linear problem solving with a focus on isogeometric multi-patch analysis and high-order finite element techniques.
	}
	\cvdatedentry{Berlin University, Research Assistant in Computational Engineering}{Apr. 2021 - Sep. 2022}{%
			Contributed to finite element software development, with a focus on local refinement techniques and high-order modeling approaches.
	}
	\cvdatedentry{Berlin University, Research Assistant in Machine Learning}{Apr. 2020 - Apr. 2021}{%
			Collaborated on an industry project for predictive maintenance using deep autoencoder networks on production data, performing software engineering tasks within a machine learning framework.
	}

	\cvsection{\faBook}{Teaching Experience}
	\cvdatedentry{Berlin University, Assistant Computational Methods}{Sep. 2022 - Present}{%
			Supported Master's-level courses in numerical techniques, including linear systems, eigenvalue problems, and finite element basics. Responsibilities included preparing problem sheets and leading consultation sessions.
	}
	\cvdatedentry{Berlin University, Introductory Engineering Project}{Winter Term 21/22}{%
			Led a practical course for first-year students, guiding them in the design and testing of a basic measurement setup for aerodynamic analysis.
	}
	\cvdatedentry{Berlin University, Tutor for Programming Fundamentals}{Oct. 2018 - Feb. 2020}{%
			Supported labs for C and Java programming courses, assisting students with coding fundamentals and project troubleshooting.
	}

	\cvsection{\faTools}{Skills}
	\cventry{Software}{%
			\textit{OS}: GNU Linux\sep macOS\sep Windows \\
			\textit{Office}: \LaTeX\sep Markdown\sep Microsoft Office Suite \\
			\textit{CAE}: ANSYS Mechanical\sep Simcenter\sep SimScale\sep MATLAB \\
			\textit{CAD}: Autodesk Inventor\sep Siemens NX\sep OnShape
	}
	\cventry{Programming}{%
			\textit{Proficient}: Python\sep MATLAB \\
			\textit{Familiar}: C\sep C\ensuremath{++}\sep Julia\sep Java
	}
	\cventry{Languages}{%
			German (native), English (C1)
	}

	\cvsection{\faBuffer}{Publications \& Talks}
	\cvdatedentry{International Conference on Computational Mechanics 2023}{Jul. 2023}{%
			Adaptive Refinement Strategies for Elastoplastic Finite Element Analysis, \textit{J. Smith, B. Schmidt, C. Mueller}
	}

	\cvsection{\faRocket}{Projects}
	\cvlinkedentry{Finite Element Solver Library}{https://github.com/johnsmith/fesolver}{github.com/johnsmith/fesolver}{%
			Developed a finite element library in Python for research purposes, focusing on high-order elements with adaptive local refinement capabilities. This project supported multiple thesis projects and incorporates sparse linear algebra.
	}

	\cvsection{\faWaveSquare}{Extracurricular Activities}
	\cvdatedentry{Formula Electric Berlin, Student Racing Team}{Oct. 2018 - Feb. 2022}{%
			Worked with a team to design and manufacture mechanical components, develop aerodynamic modules, and create software for lap time predictions based on vehicle parameters.
	}

	\cvsection{\faSkiing}{Hobbies and Interests}
	\cventry{Sports}{%
			Rock Climbing\sep Skiing\sep Cycling
	}

\end{document}
